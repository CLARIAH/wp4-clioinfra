\documentclass[a4paper]{article}
% \usepackage[a4paper]{geometry}
\usepackage[utf8]{inputenc}
\usepackage{hyperref,footmisc,eurosym}
\usepackage{multirow}
\usepackage{multicol}
\usepackage{rotating}
\usepackage{natbib,amsmath}
\usepackage{booktabs,graphicx,pgfplots,float}
\usepackage{pdflscape}
\usepackage{nameref,tabularx}
\usepackage[normalem]{ulem}
% \usepackage{lscape}
%\bibliographystyle{dinat}
%\bibliographystyle{chicago}

\linespread{1.2}
\frenchspacing
\setlength{\topmargin}{0in}
\setlength{\headheight}{0.15in}
\setlength{\headsep}{0.15in}
\setlength{\textheight}{9in}
\setlength{\textwidth}{5.5in}
\setlength{\oddsidemargin}{0.35in}
\setlength{\evensidemargin}{0in}
\setlength{\parindent}{0.15in}
\setlength{\parskip}{0.1in}
%\setlength{\voffset}{0.2in}

%%% for flow chart; start
\usepackage{tikz}
\usetikzlibrary{shapes,arrows}
\usepackage{caption}
\newcommand*{\h}{\hspace{5pt}}% for indentation
\newcommand*{\hh}{\h\h}% double indentation
%%% for flow chart; end


\title{Clio Infra Website:\\ Creation, Use and Maintenance.}
\author{Michail Moatsos\footnote{michalis.moatsos@iisg.nl, IISG}}

\begin{document}

\maketitle 
\begin{abstract}
Everything one needs to know for how the new Clio Infra website is created and 
how to recreate it from scratch if necessary. All the requirements are spelled 
out, and all materials and software used are open source. The overall approach 
uses R-coded scripts to fill-in manually made html templates, and produce the 
static pages for the web server. R code is fed with the material (both data and 
metadata) available on the 
\href{https://datasets.socialhistory.org/dataverse/clioinfra}{Clio Infra 
dataverse repository}. The html layout is manually made using 
\href{http://getbootstrap.com/}{Bootstrap}. The demo front end is reachable 
\href{clio2.sandbox.socialhistoryservices.org}{here}. Pending tickets for 
website and service improvement are available on 
\href{
https://github.com/CLARIAH/wp4-requirements/issues?q=is%3Aissue+is%3Aopen+label%
3Aclio-infra%3Awebsite}{Github}
\end{abstract}

\clearpage

\tableofcontents

\clearpage

\section{Introduction}\label{sec:intro}

%\subsection{How to read this manual}

%colorcoding...

\subsection{Built using Bootstrap}

The website is developed using Bootstrap version 3.3.7. However, some of the 
css files have been modified for additional functionalities (such as the back 
to the top button). Thus, one needs to use the Bootstrap files as they are 
found on the demo server for smooth operation. For the exact templates 
developed and used see section \ref{sec:req}.

\subsection{Workflow}

Figure \ref{fig:diagram} shows the workflow of website creation via the various 
R scripts and the available data. The Dataverse block is strikedthrough to 
underlie the fact that in the current version those files need to be fetched 
manually.

\begin{center}
\captionof{figure}{Workflow for an R generated website}\label{fig:diagram}
% setting the typeface to sans serif and the font size to small
% the scope local to the environment
\sffamily
\footnotesize
% Define block styles
\tikzstyle{decision} = [diamond, draw, fill=blue!30, 
    text width=4.5em, text badly centered, node distance=3cm, inner sep=0pt]
\tikzstyle{block} = [rectangle, draw, fill=blue!30, 
    text width=5em, text centered, rounded corners, minimum height=4em]
\tikzstyle{line} = [draw, -latex']
\tikzstyle{cloud} = [draw, ellipse,fill=red!40, node distance=3cm,
  minimum height=2em]
  
  \begin{tikzpicture}[node distance = 2cm, auto]
  % Place nodes

  \node [block] (DV) {\sout{Dataverse}};
  \node [block, below of=DV] (Data) {Data downloading};
  \node [block, below of=Data] (IList) {Indicator List};
  \node [block, below of=IList] (Rankings) {Rankings \& Performance per 
Indicator \& Country};
  \node [block, left of=Data] (LEntities) {Extraction of landed entities};
  \node [block, below of=LEntities] (OECD_UN) {OECD/UN groupings};
  \node [block, below of=OECD_UN] (CList) {Country List};
  \node [block, left of=CList] (ReGlob) {Regional and Global aggregates};
  \node [block, below of=ReGlob] (Home) {Homepage, Global, Regional pages};
  \node [block, below of=CList] (CPage) {Country pages};
  \node [block, right of=Data] (meta) {Metadata downloading (docx)};
  \node [block, right of=Rankings] (IPage) {Indicator Page};
  \node [block, right of=meta] (R) {R};
  \node [block, right of=IPage] (OtherDynamic) {News, Publications, 
Contributors};
  \node [block, below of=OtherDynamic] (OtherStatic) {Partners, About, Contact};
  \node [cloud, below of=Home] (Clio) {Github};
   \node [cloud, right of=Clio] (webserver) {webserver};
  
  % Draw edges
  
  \path [line] (DV) -- (Data);
  \path [line] (DV) -- (meta);
  \path [line] (Data) -- (IList);
  \path [line] (Data) -- (LEntities);
  \path [line] (LEntities) -- (OECD_UN);
  \path [line] (OECD_UN) -- (ReGlob);
  \path [line] (ReGlob) -- (Home);
  \path [line] (Home) -- (Clio);
  \path [line] (CPage) -- (Clio);
  \path [line] (OECD_UN) -- (CList);
  \path [line] (Rankings) -- (CList);
  \path [line] (meta) -- (IPage);
  \path [line] (IPage) -- (Clio);
  \path [line] (CList) -- (CPage);
  \path [line] (R) -- (OtherDynamic);
  \path [line] (OtherDynamic) -- (Clio);
  \path [line] (OtherStatic) -- (Clio);
  \path [line] (IList) -- (Rankings);
  \path [line] (Clio) -- (webserver);
  
  \end{tikzpicture}
\end{center}


\subsection{Overview of items generated by the scripts}

\begin{enumerate}
 \item Indicators per country in separate xlsx with long format, created by 
running CountryHTML\_JSON.R, having filename structure:\\
CountryName\_IndicatorName\_TerritorialRef\_BorderStart\_2012\_CCode\_XXX.xlsx\\
for example: 
Zambia\_LabourersRealWage\_TerritorialRef\_1964\_2012\_CCode\_894.xlsx\\
Exported at folder: /IndicatorsPerCountry.

 \item All indicators for a given country in one xlsx with wide and long 
format, created by running CountryHTML\_JSON.R, having filename structure:\\
CountryName\_AllIndicatorsAvailable\_TerritorialRef\_BorderStart\_2012\_CCode
\_XXX.xlsx\\ for example:
BosniaandHerzegovina\_AllIndicatorsAvailable\_TerritorialRef\_1992\_2012\_CCode
\_70.xlsx \\
Exported at folder: /CountryData.

 \item (a) A ``broad'' xlsx per indicator using all ``2012 border'' countries 
of the Clio Infra dataset, and (b) a ``compact'' xlsx per indicator with rows 
only for countries with available data. Both are created by using 
IndicatorHTML.R, with both wide and long formats.\\
Exported at folder: /data.

 \item One html page for each indicator\\
Exported at folder: /IndicatorPagesWithMenus.

 \item One html page for each country\\
Exported at folder: /CountryPagesWithMenus.

 \item index.html, indexOECD.html, Partners.html, News\_Publications.html 
located at the same folder as the R scripts.
  
 \item The html files that actually need to be uploaded on the server are 
located in the folder ``ToUpload''.

\item An xlsx file containing all indicators for historical entities at 
``DataAtHistoricalBorders.xlsx''.\\
Exported at folder: /data.
 
\end{enumerate}

\textbf{WARNING: if at any point the xlsx files fail to export in the r scripts 
replace write.xls with write.xls2. Do note that most of them have been already 
replaced.}

\clearpage

\section{Requirements in tools and data}\label{sec:req}

The website is created by using a set of templates and some scripts written in 
R. The templates are written in html and make use of the css functionalities of 
\href{http://getbootstrap.com/}{Bootstrap}. 

Those HTML templates are:
\begin{itemize}
 \item IndexTemplate.html
 \item IndicatorsTemplate.html
 \item CountryTemplateJSON.html and CountryTemplateNoMoreVisualsJSON.html
 \item NewsPublicationsTemplate.html
 \item PartnersTemplate.html
\end{itemize}

The R scripts are (in strict order of execution):
\begin{itemize}
 \item ReadData.R
 \item IndicatorHTML.R
 \item CountryHTML\_JSON.R
 \item HistoricalBordersData.R
 \item AddMenusToIndicatorAndHome.R
 \item AddCountryMenu.R (only run this after running 
AddMenusToIndicatorAndHome.R)
 \item Contributors.R this needs to run only once to create 
ContributorsList.xlsx
 \item NewsAndPublications.R
 \item PartnersPage.R
 \item FooterSubstitution.R
 \item some supportive scripts that only need to be called from the other 
scripts are (\textit{you don't need to call them directly}): 
IndicatorHTMLSubstitutions.R, LongToClio.R, citations.R, MakeTheGGplot.R, 
f\_IndicatorMenu.R, and f\_IndicatorMenu2.R.
\end{itemize}

Some main texts and external images used in the process are:
\begin{itemize}
 \item AboutClioInfra.txt
 \item logo of various participating entities (found in images folder under www)
\end{itemize}

To run those scripts you need to install the following R libraries: 
\textit{readxl, xlsx, ggplot2, jsonlite, tidyr, RefManageR.}

Files related to Bootstrap that need to be copied to the server www folder are 
all placed in the www folder of the Github repository.

Data files that are required (not included in the dataverse repository, but 
included in the Github repository) are:
\begin{itemize}
 \item ``CIA-Factbook-Countries with notes for their independence status.xls'', 
constructed with information found in: 
\url{
https://www.cia.gov/library/publications/the-world-factbook/fields/2088.html#af}
 \item ``statcan - countries list and codes from statistics canada.xls'', 
constructed with information found in: 
\url{http://www.statcan.gc.ca/eng/subjects/standard/sccai/2011/scountry-desc}
 \item ``UN Countries or areas, codes and abbreviations.xls'', constructed with 
information found in:
\url{http://unstats.un.org/unsd/methods/m49/m49alpha.htm}
\item ``oecdregions.csv'' file containing the OECD split of the world intro 
regions. Provided by Auke Rijpma.
\end{itemize}

\section{Processing}

The data files are not fetched automatically from the dataverse repository in 
the current version of the R scripts. This will be the case in future versions.
Thus the data (and metadata) files need to be manually placed on the proper 
folder so that R scripts can locate them. 

\subsection{Metadata files}

Until the metadata files on dataverse are updated with new ones, the pre-edited 
metadata files found on this Github repository must be fed to the R scripts 
instead. 

Because the R script requires the metadata in txt format follow the 
instructions found in ConvertDOCStoTXT.xls to convert the docx files to txt.
For convenience the txt files are also provided in the Github repository.

\subsection{The website layouts}

On the same folder level as the R scripts all the html layouts (listed in 
section \ref{sec:intro}) need to be placed.

\subsection{Exporting data}
 This takes place in \begin{verbatim}CountryHTML_JSON.R\end{verbatim} and can 
be found at the comment: 
 \begin{verbatim}
  ### Now exporting indicator and country files
 \end{verbatim}

Change the value of \textcolor{red}{ExportData} flag to export the files or 
not. If no change takes place in the data, then to save time considerably set 
this variable to \textcolor{blue}{FALSE}.

\section{Other pages}
\subsection{News \& Publications page}
This page is completely manual. When new items are added then they have to be 
written in html code within the News\_Publications.html page. This could be 
slightly improved in later versions, but the idea is not to recreate a cms 
functionality with R.

\subsection{Partners page}

Some preprocessing to extract the list of contributors is necessary. This is 
done by Contributors.R script. This needs to run only once to create 
ContributorsList.xlsx. After that only calling the PartnersPage.R is enough to 
create the html page. Then of course you need to run the FooterSubstitution.R 
script to properly set the footer.

\section{Adding new countries and indicators}

This is to explain how to update the files and scripts when one needs to add an 
indicator and countries.

First step is to place the new data and metadata files in the proper folders 
and, in case of a new indicator, convert the metadata doc file that the creator 
of the dataset has filled to a txt file. To convert the doc file to txt file 
see ConvertDOCStoTXT.xls for the commands to use in Ubuntu. Also make sure that 
the data are given in the proper layout (see file ClioLayout.xlsx), otherwise 
convert them in the proper layout. If this is not possible then additional R 
programming in necessary to digest the new format.

If the metadata are not given in the format required (see current txt files for 
the format required) then the process will not conclude. Thus, some manual 
editing of the metadata file to achieve conformity might be necessary. 
Currently the template I'm using is slightly different than the one used 
before. Thus I had to downgrade the current template to achieve conformity with 
the old one. This means that at some point I need to update the procedure that 
will allow for the new template to be used.

Second step is that the \textit{ReadData.R} script needs to be run to read the 
new data in, and then follow and execute the remaining scipts in order of 
appearance in the R script list of section \ref{sec:req}. Before running this 
script some modifications are necessary. Go to section starting with the 
comment: ``ADD WebName and WebCategory'' and add those two variables for the 
new dataset you want to include. Example:
\begin{verbatim}
 ClioMetaData$WebName[which(ClioMetaData$title=="Composite Measure of 
Wellbeing")] <- "Composite Measure of Wellbeing"
ClioMetaData$WebCategory[which(ClioMetaData$title=="Composite Measure of 
Wellbeing")] <- "Demography"
\end{verbatim}

For ease of use keep the same WebName as it is given in the original xlsx top 
cell. If not problems with naming consistency arise among the various tables in 
the scripts.

Third step requires running the IndicatorHTML.R script. Before running this 
script you need to upload the new indicator on dataverse to be able to get the 
citation info. After depositing the new indicator on dataverse you need to 
download the two types of available citations to the local citation 
folder\footnote{.../Clio Infra/Website/Citations}. Then to add the bib type of 
citation you need to either do that manually, or install ``sudo apt install 
bibutils'' on an Ubuntu system. To get the command for doing this simply paste 
the complete short description from dataverse to the third column in ``~/Clio 
Infra/Website/Citations.xls'', and then export a static version of this in 
``~/Clio Infra/Website/CitationsStatic.xls''.\footnote{Note that when working 
with the citation files from the Composite Wellbeing Index the bibutils command 
did not work, so I produced the bib file manually so that I can proceed.} 
Further, you need to add the document filename of the new indicator in the 
``~/Clio Infra/Website/IndicatorsListWithDocFiles.xlsx'', and appropriately in 
``~/Clio Infra/Website/IndicatorsGraphType.xlsx''; and in ``~/PhD/Clio 
Infra/Website/IndicPriorityList.xlsx'' set the priority in selecting the 
variable for country visualizations. 

Important for completing successfully the procedure: the consistent naming of 
the indicator in various locations is critical. The filename of the xlsx with 
the original data must end with ``-historical.xslx''. The prefix (e.g. 
CompositeWellbeingIndex) should be used as it is in the docx and the txt files. 
It is preferable that the prefix will also be used in the ``ADD WebName and 
WebCategory'' above. Very important: the WebName must be used in the 
Citations.xlsx list above.

Running ReadData.R will execute in about 2 minutes on a good laptop, and 
IndicatorHTML.R will do so in approximately 10 minutes (with all flags marked 
as True). CountryHTML\_JSON.R takes about 35 minutes with all flags up.

In the section that begins with ``\# PARAMETERS:'' in scropt CountryHTML\_JSON.R
you can set the parameters with which the data for the visualizations in the 
country HTML pages will be selected.

Note that when adding a historical entity a different process needs to be 
followed in script: AddMenusToIndicatorAndHome.R

To avoid reproducing the historical data files set CreateHistoricalDataFiles to 
false in script: AddMenusToIndicatorAndHome.R

\section{Uploading to the server}

\subsection{Folder correspondence table}

\begin{table}[h]
\centering
\caption{Where on server to upload which items from which local folder.}
\label{tab:uncertainty}
\begin{tabular}{l | l}
  \toprule
  Local Folder & Server Folder (under /var/www) \\
  \midrule
  ToUpload (no subdirectories) & / \\
  ToUpload/CountryPagesWithMenus & /Countries \\
  ToUpload/IndicatorPagesWithMenus & /Indicators \\
  Country data & /docs \\
  JSON & /json \\
  IndicatorsPerCountry & /IndicatorsPerCountry \\
  data & /data \\
  html/graphs & /graphs \\
  Citations & /Citations \\
  \bottomrule
\end{tabular}
\end{table}

\textbf{Do note that the docx and xlsx files with the original data as deposited 
by the contributors must also be placed in the /docs folder of the server}

\subsection{Commands to upload to initial test server}
\begin{verbatim}
ssh michalism@clio2.sandbox.socialhistoryservices.org

scp -r "/var/www/html/theme.css" 
michalism@clio2.sandbox.socialhistoryservices.org:/home/michalism
sudo cp theme.css /var/www/

scp -r "/home/michalis/PhD/Clio Infra/Website/CountryPagesWithMenus" 
michalism@clio2.sandbox.socialhistoryservices.org:/home/michalism
sudo cp CountryPagesWithMenus/* /var/www/Countries/
scp -r "/home/michalis/PhD/Clio Infra/Website/JSON" 
michalism@clio2.sandbox.socialhistoryservices.org:/home/michalism
sudo cp JSON/* /var/www/json/
scp -r "/home/michalis/PhD/Clio Infra/Website/IndicatorsPerCountry" 
michalism@clio2.sandbox.socialhistoryservices.org:/home/michalism
sudo cp -R IndicatorsPerCountry/ /var/www/

scp "/home/michalis/PhD/Clio Infra/Website/index.html" 
michalism@clio2.sandbox.socialhistoryservices.org:/home/michalism
sudo cp index.html /var/www/
scp -r "/home/michalis/PhD/Clio Infra/Website/html/graphs" 
michalism@clio2.sandbox.socialhistoryservices.org:/home/michalism
sudo cp -R graphs/ /var/www/
scp -r "/home/michalis/PhD/Clio Infra/Website/IndicatorPagesWithMenus" 
michalism@clio2.sandbox.socialhistoryservices.org:/home/michalism
sudo cp -R IndicatorPagesWithMenus/* /var/www/Indicators/
scp -r "/home/michalis/PhD/Clio Infra/Website/Citations" 
michalism@clio2.sandbox.socialhistoryservices.org:/home/michalism
sudo cp -R Citations/* /var/www/citations/
scp -r "/home/michalis/PhD/Clio Infra/Website/data" 
michalism@clio2.sandbox.socialhistoryservices.org:/home/michalism
sudo cp -R data/ /var/www/
scp -r "/home/michalis/PhD/Clio Infra/Website/CountryData" 
michalism@clio2.sandbox.socialhistoryservices.org:/home/michalism
sudo cp -R CountryData/* /var/www/docs/

scp "/var/www/html/index.html" 
michalism@clio2.sandbox.socialhistoryservices.org:/home/michalism
scp -r "/var/www/html/images" 
michalism@clio2.sandbox.socialhistoryservices.org:/home/michalism
sudo cp -R images/* /var/www/images/
\end{verbatim}

%\section{Bibliography}

%\bibliographystyle{apalike}
%\bibliography{library.bib}

\section{Appendix}
\subsection{Functions and Scripts}
\subsubsection{LongToClio.R}
This function converts long format to clio infra format
\subsection{MadToClio.R}
This function converts the format used in the Maddison project, to that
used in the clio-infra website and dataverse. This function was written when 
adding the Composite Measure of Wellbeing, and it is written around it, thus it 
is not a generic script yet.
\subsubsection{Contributors.R}
Exports the list of authors with their homepage and affiliation to 
ContributorsList.xlsx. The correct path needs to be provided.
\end{document}
